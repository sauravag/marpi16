\documentclass[12pt]{article}
\usepackage{amsmath}
\usepackage{color}

\newcommand{\cmt}[1]{\textcolor{red}{//TODO: #1}}


\title{State Estimation for Pancake Robot: \\ \small{Derivation of Theta From Gravity, Robot Acceleration and $\phi$}}

\author{Nicholas Hemstreet}

\date{}

\begin{document}

\maketitle

\section{Introduction}

\cmt{Fill it out}

\section{Preliminaries}

Let $x_k$ be the robot state at time $t_k$ defined as 

\begin{equation}
\mathbf{x}_k = 
\begin{bmatrix}
	\phi_k \\
	\theta_k \\
	d_k
\end{bmatrix},
\end{equation}

where $\phi_k$ is the angle that body x-axis makes with the flow direction, $\theta_k$ is the circumferential position of the robot and $d_k$ is the linear displacement along the pipe. Let $u_k$ be the control input (what we are applying to the system, for our case the velocity and angular velocity of the robot) 

\begin{equation}\label{eq:control-vector}
\mathbf{u}_k = 
\begin{bmatrix}
	v_k \\
	\omega_k
\end{bmatrix}.
\end{equation}  

Let $w_k \sim \mathcal{N}(0, \mathbf{Q}_k)$ be the zero-mean Gaussian process noise such that 

\begin{equation}\label{eq:control-noise}
	\mathbf{w}_k = 
	\begin{bmatrix}
		n_{v_k} \\
		n_{\omega_k} \\
	\end{bmatrix}.
\end{equation}

Given the state, control and process noise, state transition function or motion model is defined as $f(x_k, u_k,w_k)$. \\

We define the observation $z_k$, and observation model $h(x_k, \nu_k)$ such that $z_{k} = h(x_k, \nu_k)$ where $\nu_k \sim \mathcal{N}(0, \mathbf{R}_k)$ is zero-mean Gaussian observation noise. The robot belief at time $t_k$ is $b_k \sim \mathcal{N}(\hat{\mathbf{x}}^{+}_k, \mathbf{P}^{+}_k)$ where $\mathbf{x}^{+}_k$ is the robot state estimate at $t_k$ and $\mathbf{P}^{+}_k$ is the state error covariance.

\section{Motion Model}\label{sec:motion-model}

We can now define the state transition function or motion model $f(x_k, u_k,w_k)$ that takes $x_k$ (the state variable), a control input $u_k$ (what we are applying to the system, for our case the velocity and angular velocity of the robot) and $w_k$ (the noise in each of the elements) such that :

\begin{equation}\label{eq:motion-model}
\mathbf{f}(x_k,u_k,w_k) = 
\begin{bmatrix}
 	\phi_k + \omega_k \Delta T + n_{\omega_k} \sqrt{\Delta T} \\
 	\theta_k + \frac{v_k  \sin(\phi_k)\Delta T}{R} + n_{v_k} \frac{\sin(\phi_k)\sqrt{\Delta T}}{R} \\
 	d_k + v_k\cos(\phi_k)\Delta T + n_{v_k} \cos(\phi_k)  \sqrt{\Delta T}
\end{bmatrix}
\end{equation} 

With the motion model function defined we need to find the Jacobians so that we can build the Kalman Filter. \\

We define the motion model jacobian w.r.t state as 

\begin{equation}
	\mathbf{F}_{k} = \bar{\nabla}_x\mathbf{f} = 
	\begin{bmatrix}
		1 & 0 & 0 \\
		v_k \frac{\cos(\phi_k)}{R}\Delta T & 1 & 0 \\
		-v_k\sin(\phi)\Delta T & 0 & 1\\
	\end{bmatrix}
\end{equation} 

and w.r.t noise as 

\begin{equation}
\mathbf{L}_{k} = \bar{\nabla}_w\mathbf{f} = 
\begin{bmatrix}
	0 & \sqrt{\Delta T} \\
	\frac{\sin(\phi_k) \sqrt{\Delta T}}{R} & 0 \\
	\cos(\phi_k)\sqrt{\Delta T} & 0 \\
\end{bmatrix}.	 
\end{equation} 

\section{Observation Model}\label{sec:obs-model}

We have the observation at time $t_k$ as

\begin{equation}
	\mathbf{z}_k = 
		\begin{bmatrix}
			a_x \\
			a_y \\
			a_z \\
			\phi_k 
		\end{bmatrix}
		=
	\begin{bmatrix}
		\mathbf{R}^T_{bp}\mathbf{g} \\
		\phi_k 
	\end{bmatrix},
\end{equation}

where $\mathbf{R}_{bp}$ is the rotation matrix from body frame to pipe frame and $\mathbf{g}$ is the gravity vector in pipe (inertial) frame. The gravity vector in pipe frame can be written out as

\begin{equation}
\mathbf{g} = 
\begin{bmatrix}
-g \\
 0\\
 0
\end{bmatrix},
\end{equation}

and $\mathbf{R}_{bp}$ as

\begin{equation}
 R_{bp} = 
 \begin{bmatrix}
 \sin(\theta)\sin(\phi) & -\sin(\theta)\cos(\phi) & -\cos(\theta) \\
 -\cos(\theta)\sin(\phi)& \cos(\theta)\cos(\phi)  & -\sin(\theta) \\
 \cos(\phi)             & \sin(\phi)              & 0 \\
 \end{bmatrix}.
\end{equation}

Now we may define the observation as

\begin{equation}
	\mathbf{z}_k = 
	\begin{bmatrix}
	-g \sin(\theta_k) \sin(\phi_k) + \nu_{k,a_x} \\
	g \sin(\theta_k) \cos(\phi_k) + \nu_{k,a_y} \\
	g \cos(\theta_k) + \nu_{k,a_z} \\
	\phi_k + \nu_{k,\phi}
	\end{bmatrix}.
\end{equation}

With the observation function defined we need to find the Jacobians so that we can build the Kalman Filter. \\

The observation jacobian w.r.t state is 

\begin{equation}
	\mathbf{H}_k = \bar{\nabla}_x\mathbf{h} =
	\begin{bmatrix}
	-g \sin(\theta_k) \cos(\phi_k) & -g \cos(\theta_k) \sin(\phi_k) & 0 \\
	-g \sin(\theta_k) \sin(\phi_k) & g \cos(\theta_k) \cos(\phi_k) & 0 \\
	 0 & -g \sin(\theta_k) & 0 \\
	 1 & 0 & 0
	\end{bmatrix}
\end{equation}

The observation jacobian w.r.t noise is 

\begin{equation}
	\mathbf{M}_k = \bar{\nabla}_{\nu}\mathbf{h} = \mathbf{I}_{4 \times 4}
\end{equation}

where $ \mathbf{I}_{4 \times 4}$ is identity matrix of size $4 \times 4$.

\section{Noise Covariance Matrices}

At time $t_0$ the belief is  $b_k \sim \mathcal{N}(\hat{\mathbf{x}}^{+}_0, \mathbf{P}^{+}_0)$.

\begin{equation}
\mathbf{P}^{+}_0 =
\begin{bmatrix}
0.0001 & 0 & 0 \\
0 & 0.0001 & 0 \\
0 & 0 & 0.0001
\end{bmatrix}
\end{equation}

\begin{equation}
\mathbf{Q}_k =
\begin{bmatrix}
0.0001 & 0\\
0 & 0.0001
\end{bmatrix}
\end{equation}

\begin{equation}
\mathbf{R}_k =
\begin{bmatrix}
0.0001 & 0 & 0 & 0\\
0 & 0.0001 & 0 & 0\\
0 & 0 & 0.0001 & 0\\
0 & 0 & 0 & 0.0001 \\
\end{bmatrix}
\end{equation}

\end{document}